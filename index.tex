%  --------------------------------------------------------------------------
%  Semesterarbeit Single Sign-On Lösung für Django Dokumentation
%  Created by Marc Egli on 2012-04-02.
%  --------------------------------------------------------------------------

%  --------------------------------------------------------------------------
%  Latex Document Settings
%  --------------------------------------------------------------------------
\documentclass[
11pt, % Schriftgrösse
a4paper, % A4 Papier
BCOR10mm, % Absoluter Wert der Bindekorrektur, z.B. BCOR1cm
DIV14, % Satzspiegel festlegen siehe
       % http://www.ctex.org/documents/packages/nonstd/koma-script.pdf
footsepline = false, % Trennlinie zwischen Textkörper und Fußzeile
                     % bei normalen Seiten
headsepline, % Trennlinie zwischen Kopfzeile und Textkörper
             % bei normalen Seiten
oneside, % Zweiseitig
openright,
halfparskip, % Europäischer Satz mit Abstand zwischen den Absätzen
abstracton, % inkl. Abstract
listof=totocnumbered, % Abb.- und Tab.verzeichnis im Inhaltsverzeichnis
bibliography=totocnumbered % Lit.zeichnis in Inhaltsverzeichnis aufnehmen
]{scrreprt}

\usepackage[automark]{scrpage2} % Gestaltung von kopf- und Fußzeile
\usepackage[ngerman]{babel}
\usepackage[ngerman]{translator}
\usepackage{tocbasic}
\usepackage[utf8]{inputenc}
\usepackage{lmodern} % Latin Modern
\usepackage[T1]{fontenc}
\usepackage{hyphenat}
\usepackage{ae} % Schöne Schriften für PDF-Dateien
\usepackage{multirow} % tabellenzellen zusammenfassen
% \usepackage{slashbox} % Tabellenbox welche diagonal getrennt ist

% Tradmark
\def\TTra{\textsuperscript{\texttrademark}}

%1.5 Zeilenabstand
\usepackage[onehalfspacing]{setspace}

% Festlegung des Seitenstils (scrpage2)
\pagestyle{scrheadings}
\clearscrheadfoot
\automark[chapter]{section}

% \lehead{\sffamily\upshape\headmark}
% \cehead{}
% \rehead{}
% \lefoot[\pagemark]{\upshape \pagemark}
% \cefoot{}
% \refoot{}
% \lohead{}
% \cohead{}
\lohead{\sffamily\upshape\headmark}
\lofoot{}
\cofoot{}
\rofoot[\pagemark]{\scshape \pagemark}

% Surround parts of graphics with box
\usepackage{boxedminipage}

% Package for including code in the document
\usepackage{listings}
\lstset{language=Python}

% If you want to generate a toc for each chapter (use with book)
\usepackage{minitoc}
\usepackage{longtable}

% Abkürzungsverzeichnis erstellen.
\usepackage[printonlyused]{acronym}

% schöne Tabelle zeichnen
\usepackage{booktabs}
\renewcommand{\arraystretch}{1.4} %Die Zeilenabstände in Tabllen angepasst.

% für variable Breiten
\usepackage{tabularx}

% Durchgestrichener Text
\usepackage[normalem]{ulem} %emphasize weiterhin kursiv

% This is now the recommended way for checking for PDFLaTeX:
\usepackage{ifpdf}

\usepackage{eurosym}

\usepackage{natbib}

\usepackage{paralist}

\usepackage{array,ragged2e}

% glossary
\usepackage[xindy,acronym]{glossaries}

\usepackage[]{hyperref}
\hypersetup{
  bookmarks=true,         % show bookmarks bar?
  unicode=true,           % non-Latin characters in Acrobat’s bookmarks
  pdftoolbar=true,        % show Acrobat’s toolbar?
  pdfmenubar=true,        % show Acrobat’s menu?
  pdffitwindow=true,      % window fit to page when opened
  pdfstartview={FitH},    % fits the width of the page to the window
  pdftitle={Semesterarbeit},   
  pdfauthor={Marc Egli},
  pdfsubject={Einzelrechnersuchmaschine im Vergleich mit Google},
  pdfcreator={TeX Live 2011},
  pdfproducer={pdfTeX, Version 3.1415926-2.3-1.40.12},
  pdfnewwindow=true,      % links in new window
  colorlinks=true,       % false: boxed links; true: colored links
  linkcolor=blue,          % color of internal links
  citecolor=black,        % color of links to bibliography
  filecolor=magenta,      % color of file links
  urlcolor=cyan          % color of external links
  % linkcolor=black,          % color of internal links
  % citecolor=black,        % color of links to bibliography
  % filecolor=black,      % color of file links
  % urlcolor=black          % color of external links
}

\ifpdf
    \usepackage[pdftex]{graphicx}
\else
    \usepackage{graphicx}
\fi


\usepackage{semantic}
\usepackage{syntax}
\AtBeginDocument{\catcode`\_=8 }

\makeatletter 
\let\orgdescriptionlabel\descriptionlabel 
\renewcommand*{\descriptionlabel}[1]{% 
  \let\orglabel\label 
  \let\label\@gobble 
  \phantomsection 
  \edef\@currentlabel{#1}% 
  %\edef\@currentlabelname{#1}% 
  \let\label\orglabel 
  \orgdescriptionlabel{#1}% 
} 
\makeatother 

\include{glossary}
\makeglossaries

%  --------------------------------------------------------------------------
%  Start Document
%  --------------------------------------------------------------------------
\title{Domain Specific Language}

\author{Seminararbeit Syntax und Semantik\\
    \\
    Studierender - Marc Egli\\
    Projektbetreuer - Beat Seeliger\\
	\\
	ZHAW}

\date{März 2013 bis Juni 2013}


\begin{document}
  \ifpdf
    \DeclareGraphicsExtensions{.pdf, .jpg, .tif}
  \else
    \DeclareGraphicsExtensions{.eps, .jpg}
  \fi

  \pagenumbering{Alph}
  
  \maketitle

  \pagenumbering{Roman}
  
  \tableofcontents
  
  \pagenumbering{arabic}
  
  \chapter{Aufgabenstellung}
  \label{cha:Aufgabenstellung}
  %!TEX root = ../index.tex
\section{Ausgangslage}
\label{sec:Ausgangslage}
Die Webseite \url{http://frog32.ch} enthält eine Komponente, welche in Echtzeit allen Benutzern die Mauszeiger-Position sämtlicher Benutzer anzeigt. Dieser Echtzeitdienst wurde mit dem Twisted Framework erstellt und besitzt einen eingebauten Jabber Bot, über welchen zur Entwicklung notwendige Daten bezogen werden können und sich der Dienst steuern lässt.

\section{Ziel der Arbeit}
\label{sec:Ziel der Arbeit}
Der erwähnte Jabber Bot versteht zur Zeit nur zwei Befehle. Für eine einfachere Integration zukünftiger Funktionalität sollen diese Befehle durch eine Domain Specific Language ersetzt werden. Mit dieser Sprache soll eine direkte Interaktion mit bestimmten Variablen des Dienstes möglich sein.

\section{Aufgabenstellung}

\label{sec:Aufgabenstellung}
Es soll eine DSL entworfen werden, welche folgenden Ansprüchen genügt:

\begin{enumerate}
  \item Es soll die Möglichkeit bestehen, Systemvariablen zu registrieren, welche dann in der DSL benutzt werden können.
  \item Die Syntax für Schleifen und Bedingungen soll ohne ``end'' Tokens auskommen. Als Beispiel soll die Programmiersprache Python dienen.
  \item Die Sprache soll so aufgebaut sein, dass sie in einer interaktiven Shell zum Einsatz kommen kann.
\end{enumerate}

Zu dieser DSL soll ein Interpreter programmiert werden, welcher diese Sprache ausführen kann.

\section{Erwartete Resultate}
\label{sec:Erwartete Resultate}

\begin{itemize}
  \item DSL mit einer Beschreibung der Syntax
  \item Integration der DSL in einen Prototyp der Webseite
  \item Technischer Kurzbericht
\end{itemize}

\section{Geplante Termine}
\label{sec:Geplante Termine}
Präsentation ZHAW

  \chapter{Design}
  \label{cha:Design}
  \section{PLY}
\label{sec:ply}
Um den Lexer und den Parser zu implementieren, wurde PLY~\footnote{\url{http://www.dabeaz.com/ply/}} verwendet. PLY ist eine Impementation von Lex und Yacc in Python. Es wird jeweils zuerst ein Lexer konfiguriert, welcher die Sprache in einzelne Tokens unterteilt. Da die Sprache Blöcke anhand des Zeileneinzuges erkennt, musste der Parser so modifiziert werden, dass er bei einer Vergrösserung des Einzuges ein zusätzliches ``START\_BLOCK'' Token und bei einer Verringerung ein ``END\_BLOCK'' Token generiert.

Der Lexer erzeugt einen Stream von Tokens, auf welchen danach der Parser arbeitet. Er versucht die Ableitungsregeln, welche er als BNF Notation bekommt auf den Stream anzuwenden und so einen sinnvollen Parsebaum zu erzeugen.

\begin{lstlisting}
def p_if_statement(p):
    '''if_statement : IF expr NEWLINE START_BLOCK statement_list END_BLOCK'''
    p[0] = language.If(p[2], p[5])
\end{lstlisting}

Mit PLY werden Regeln für den Parser als Funktion umgesetzt. Der Docstring der Funktion enthält jeweils die Regel in einer BNF ähnlichen Notation. Falls die Regel eingesetzt wird, wird die Funktion aufgerufen. Der einzige Parameter `p' ist indexierbar und enthält alle Symbole welche auf der rechten Seite der Regel stehen in den Indizes 1 bis n. In den Index 0 sollte dann das Resultat geschrieben werden.

\section{Interpreter}
\label{sec:interpreter}
Der Interpreter enthält den zu interpretierenden Programmcode als String. Er verwendet den Parser um daraus einen Baum aus Objekten zu erstellen. Auf dem Objekt an der Wurzel des Baumes führt der Interpreter dann die Methode ``evaluate'' aus und gibt ihr ein Environment mit. Auf diesem Environment wird dann der Programmcode ausgeführt.

\section{Environment}
\label{sec:environment}
Das Environment auf dem der Interpreter ein Script ausführt, kann im einfachsten Fall ein assoziatives Array sein. Für einige Funktionen der Sprache ist es aber von Vorteil, wenn das Environment wie ein Stack verschiedene Kontexte übereinander stapeln kann. Weiterhin bietet das implementierte Environment die Möglichkeit Systemvariabeln zu erstellen.
  
  \chapter{Sprachdefinition}
  \label{cha:sprachdefinition}
  \section{Terminal Symbole}
\label{sec:terminal_symbole}
Da die Sprache Blöcke durch die Anzahl führender Leerzeichen auf einer Programmzeile erkennt, mussten die zwei Terminal Symbole ``START\_BLOCK'' und ``END\_BLOCK'' eingeführt werden. Diese werden durch den Lexer automatisch eingefügt, wenn sich die Anzahl Lehrzeichen verändert. Weiterhin hat die Sprache reservierte Schlüsselwörter welche in Tabelle~\ref{tab:reserved_keywords} aufgeführt sind.

\begin{table}[h!]
  \centering
  \begin{tabular}{ll}
  \toprule
    Zeichen & Symbolbezeichnung\\
  \midrule
    if & IF \\
  \hline
    nop & NOP \\
  \hline
    print & PRINT \\
  \hline
    for & FOR \\
  \hline
    in & IN \\
  \bottomrule
  \end{tabular}
  \caption{Reservierte Schlüsselwörter}
  \label{tab:reserved_keywords}
\end{table}

Weiterhin wurden die in Tabelle~\ref{tab:single_char_terminals} ersichtlichen terminal Symbole definiert welche jeweils nur aus einem Zeichen bestehen.
Diese terminal Symbole werden in den folgenden BNF Darstellungen jeweils durch ihre Zeichenrepresentation und nicht durch den Symbolnamen ausgedrückt. 

\begin{table}[h!]
  \centering
  \begin{tabular}{ll}
  \toprule
    Zeichen & Symbolbezeichnung\\
  \midrule
    . & DOT \\
  \hline
    + & PLUS \\
  \hline
    - & MINUS \\
  \hline
    * & TIMES \\
  \hline
    / & DIVIDE \\
  \hline
    ( & LPAREN \\
  \hline
    ) & RPAREN \\
  \hline
    [ & LSPAREN \\
  \hline
    ] & RSPAREN \\
  \hline
    , & COMMA \\
  \hline
    \textbackslash n & NEWLINE \\
  \bottomrule
  \end{tabular}
  \caption{Einzeichen terminal Symbole}
  \label{tab:single_char_terminals}
\end{table}

Zusätzlich zu den erwähnten Terminalsymbolen gibt es noch verschiedene Zuweisungsoperatoren (Tabelle~\ref{tab:assign}) und Symbole wie ``STRING'', ``NUMBER'' oder ``NAME'' welche vom Lexer erkannt werden.

\begin{table}[h!]
  \centering
  \begin{tabular}{ll}
  \toprule
    Zeichen & Bedeutung\\
  \midrule
    = & eifache Zuweisung \\
  \hline
    += & Addition zu einer Variablen \\
  \hline
    -= & Subtraktion von einer Variablen \\
  \hline
    *= & Multiplikation einer Variablen \\
  \hline
    /= & Division einer Variablen \\
  \bottomrule
  \end{tabular}
  \caption{ASSIGN Symbole}
  \label{tab:assign}
\end{table}

\pagebreak

\section{Nicht terminal Symbole}
\label{sec:nicht_terminal_symbole}
\subsubsection{Statement und Statement Listen}
\label{ssub:statement_und_statement_listen}
\begin{grammar}
<statement\_list> ::= <statement\_list> `\\n' <statement>
\alt <statement>
\end{grammar}

Das äusserste Element nach dem Parsen eines Scripts ist immer ``statement\_list''. Eine ``statement\_list'' besteht aus einem oder mehreren ``statement''. Bei mehreren sind diese jeweils durch einen Zeilenschlag voneinander getrennt.

\begin{grammar}
<statement> ::= <assignment>
\alt <if_statement>
\alt <for_statement>
\alt <nop>
\alt <print>  
\end{grammar}

Ein Statement besteht jeweils aus genau einem weiteren nicht terminal Symbol.

\subsection{Zuweisung}
\label{sub:zuweisung}

\begin{grammar}
<assignment> ::= <variable> <ASSIGN> <expr>
\end{grammar}

Variablen müssen nicht explizit initialisiert werden. Durch eine Zuweisung kann eine Variable implizit erstellt werden. Globale Variablen können ``Read only'' sein. Falls man versuch auf sie zu schreiben wird eine Fehlermeldung ausgegeben.

\subsubsection{Beispiel}

\begin{lstlisting}
a = b
\end{lstlisting}

\subsection{for-Schleife}
\label{sub:for_schleife}

\begin{grammar}
  <for\_statement> ::= `for' <NAME> `in' <expr> `\\n' <START\_BLOCK> <statement\_list> <END\_BLOCK>
\end{grammar}

\subsubsection{Beispiel}

\begin{lstlisting}
for a in b
    print a
\end{lstlisting}

\subsection{If-Bedinung}
\label{sub:if_bedinung}

\begin{grammar}
<if-statement> ::= `if' <expr> `\\n' <START_BLOCK> <statement_list> <END_BLOCK>
\end{grammar}

\subsubsection{Beispiel}
\begin{lstlisting}
if a == 1
    print "Bedingung erfuellt"
\end{lstlisting}

\subsection{Listen}
\label{sub:listen}
\begin{grammar}
<list> ::= `[' <list_inner> `]'
\end{grammar}
\begin{grammar}
<list\_inner> ::= <list\_part>
\alt <list\_inner> `,' <list\_part>
\end{grammar}
\begin{grammar}
<list\_part> ::= <NUMBER>
\alt <STRING>
\alt <variable>
\end{grammar}

Listen verhalten sich wie Arrays und können nur Ganzzahlen beinhalten.

\begin{lstlisting}
  a = [1, 2, 3]
\end{lstlisting}

\subsection{Mathematische Ausdrücke}
\label{sub:mathematische_ausdrücke}

\begin{grammar}
<expr> ::= <term>
\alt <expr> `+' <term>
\alt <expr> `-' <term>
\end{grammar}

\begin{grammar}
<term> ::= <term> `*' <factor>
\alt <term> `/' <factor>
\alt <factor>
\end{grammar}

\begin{grammar}
<factor> ::= <NUMBER>
\alt <STRING>
\alt <list>
\alt <variable>
\alt <result>
\alt `(' <expr> `)'
\end{grammar}

Um die korrekte Auswertung eines mathematischen Ausdrucks gemäss der Präzedenz der verwendeten Operatoren zu erreichen, wurde hier ein dreistufiger Aufbau von Regeln verwendet. So besteht das Symbol ``expr'' immer aus einer Addition oder Subtraktion von einer ``expr'' mit einem ``term'' oder nur aus einem ``term''. ``term'' wiederum besteht aus einer Multiplikation oder Division eines ``term'' mit einem ``factor'' oder nur einem ``factor''. Ein ``factor'' hingegen kann aus einem ``string'', ``list'', ``variable'', ``result'' oder wiederum einer ``expr'' welche in Klammern steht bestehen.

\begin{grammar}
<nop> ::= `nop'
\end{grammar}

\begin{grammar}
<variable> ::= <NAME>
\alt <variable> `.' <NAME>
\alt <variable> `[' <expr> `]'
\end{grammar}

\begin{grammar}
<result> ::= <variable> `(' <list\_inner> `)'
\end{grammar}

\begin{grammar}
<print> ::= `print' <expr>
\end{grammar}

  \chapter{Fazit}
  \label{cha:Fazit}
  %!TEX root = ../index.tex

Seitdem ich an der Djangoconeu 2012 den Talk von Matthieu Amiguet über Domain specific Languages gehört habe, wusste ich, dass ich dies eines Tages selber ausprobieren möchte. Da ich kurz vor dem Beginn dieses Seminars einen Jabber Bot für meine Webseite geschrieben habe, welcher grundlegende Befehle wie ``Benutzer auflisten'' und ``Benutzer umbenennen'' verstand, war für mich naheliegend hier eine DSL zu verwenden.

Eine DSL zu kreieren war eine spannende Herausforderung, welche mir grossen Spass bereitete. Ich konnte mich ausgiebig mit den Problemen, die das Design einer Sprache mit sich bringt, auseinandersetzen.


\end{document}
