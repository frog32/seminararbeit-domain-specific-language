\section{Terminal Symbole}
\label{sec:terminal_symbole}
Da die Sprache Blöcke durch die Anzahl führender Leerzeichen auf einer Programmzeile erkennt, mussten die zwei Terminal Symbole ``START\_BLOCK'' und ``END\_BLOCK'' eingeführt werden. Diese werden durch den Lexer automatisch eingefügt, wenn sich die Anzahl Lehrzeichen verändert. Weiterhin hat die Sprache reservierte Schlüsselwörter welche in Tabelle~\ref{tab:reserved_keywords} aufgeführt sind.

\begin{table}[h!]
  \centering
  \begin{tabular}{ll}
  \toprule
    Zeichen & Symbolbezeichnung\\
  \midrule
    if & IF \\
  \hline
    nop & NOP \\
  \hline
    print & PRINT \\
  \hline
    for & FOR \\
  \hline
    in & IN \\
  \bottomrule
  \end{tabular}
  \caption{Reservierte Schlüsselwörter}
  \label{tab:reserved_keywords}
\end{table}

Weiterhin wurden die in Tabelle~\ref{tab:single_char_terminals} ersichtlichen terminal Symbole definiert welche jeweils nur aus einem Zeichen bestehen.
Diese terminal Symbole werden in den folgenden BNF Darstellungen jeweils durch ihre Zeichenrepresentation und nicht durch den Symbolnamen ausgedrückt. 

\begin{table}[h!]
  \centering
  \begin{tabular}{ll}
  \toprule
    Zeichen & Symbolbezeichnung\\
  \midrule
    . & DOT \\
  \hline
    + & PLUS \\
  \hline
    - & MINUS \\
  \hline
    * & TIMES \\
  \hline
    / & DIVIDE \\
  \hline
    ( & LPAREN \\
  \hline
    ) & RPAREN \\
  \hline
    [ & LSPAREN \\
  \hline
    ] & RSPAREN \\
  \hline
    , & COMMA \\
  \hline
    \textbackslash n & NEWLINE \\
  \bottomrule
  \end{tabular}
  \caption{Einzeichen terminal Symbole}
  \label{tab:single_char_terminals}
\end{table}

Zusätzlich zu den erwähnten Terminalsymbolen gibt es noch verschiedene Zuweisungsoperatoren (Tabelle~\ref{tab:assign}) und Symbole wie ``STRING'', ``NUMBER'' oder ``NAME'' welche vom Lexer erkannt werden.

\begin{table}[h!]
  \centering
  \begin{tabular}{ll}
  \toprule
    Zeichen & Bedeutung\\
  \midrule
    = & eifache Zuweisung \\
  \hline
    += & Addition zu einer Variablen \\
  \hline
    -= & Subtraktion von einer Variablen \\
  \hline
    *= & Multiplikation einer Variablen \\
  \hline
    /= & Division einer Variablen \\
  \bottomrule
  \end{tabular}
  \caption{ASSIGN Symbole}
  \label{tab:assign}
\end{table}

\pagebreak

\section{Nicht terminal Symbole}
\label{sec:nicht_terminal_symbole}
\subsubsection{Statement und Statement Listen}
\label{ssub:statement_und_statement_listen}
\begin{grammar}
<statement\_list> ::= <statement\_list> `\\n' <statement>
\alt <statement>
\end{grammar}

Das äusserste Element nach dem Parsen eines Scripts ist immer ``statement\_list''. Eine ``statement\_list'' besteht aus einem oder mehreren ``statement''. Bei mehreren sind diese jeweils durch einen Zeilenschlag voneinander getrennt.

\begin{grammar}
<statement> ::= <assignment>
\alt <if_statement>
\alt <for_statement>
\alt <nop>
\alt <print>  
\end{grammar}

Ein Statement besteht jeweils aus genau einem weiteren nicht terminal Symbol.

\subsection{Zuweisung}
\label{sub:zuweisung}

\begin{grammar}
<assignment> ::= <variable> <ASSIGN> <expr>
\end{grammar}

Variablen müssen nicht explizit initialisiert werden. Durch eine Zuweisung kann eine Variable implizit erstellt werden. Globale Variablen können ``Read only'' sein. Falls man versuch auf sie zu schreiben wird eine Fehlermeldung ausgegeben.

\subsubsection{Beispiel}

\begin{lstlisting}
a = b
\end{lstlisting}

\subsection{for-Schleife}
\label{sub:for_schleife}

\begin{grammar}
  <for\_statement> ::= `for' <NAME> `in' <expr> `\\n' <START\_BLOCK> <statement\_list> <END\_BLOCK>
\end{grammar}

\subsubsection{Beispiel}

\begin{lstlisting}
for a in b
    print a
\end{lstlisting}

\subsection{If-Bedinung}
\label{sub:if_bedinung}

\begin{grammar}
<if-statement> ::= `if' <expr> `\\n' <START_BLOCK> <statement_list> <END_BLOCK>
\end{grammar}

\subsubsection{Beispiel}
\begin{lstlisting}
if a == 1
    print "Bedingung erfuellt"
\end{lstlisting}

\subsection{Listen}
\label{sub:listen}
\begin{grammar}
<list> ::= `[' <list_inner> `]'
\end{grammar}
\begin{grammar}
<list\_inner> ::= <list\_part>
\alt <list\_inner> `,' <list\_part>
\end{grammar}
\begin{grammar}
<list\_part> ::= <NUMBER>
\alt <STRING>
\alt <variable>
\end{grammar}

Listen verhalten sich wie Arrays und können nur Ganzzahlen beinhalten.

\begin{lstlisting}
  a = [1, 2, 3]
\end{lstlisting}

\subsection{Mathematische Ausdrücke}
\label{sub:mathematische_ausdrücke}

\begin{grammar}
<expr> ::= <term>
\alt <expr> `+' <term>
\alt <expr> `-' <term>
\end{grammar}

\begin{grammar}
<term> ::= <term> `*' <factor>
\alt <term> `/' <factor>
\alt <factor>
\end{grammar}

\begin{grammar}
<factor> ::= <NUMBER>
\alt <STRING>
\alt <list>
\alt <variable>
\alt <result>
\alt `(' <expr> `)'
\end{grammar}

Um die korrekte Auswertung eines mathematischen Ausdrucks gemäss der Präzedenz der verwendeten Operatoren zu erreichen, wurde hier ein dreistufiger Aufbau von Regeln verwendet. So besteht das Symbol ``expr'' immer aus einer Addition oder Subtraktion von einer ``expr'' mit einem ``term'' oder nur aus einem ``term''. ``term'' wiederum besteht aus einer Multiplikation oder Division eines ``term'' mit einem ``factor'' oder nur einem ``factor''. Ein ``factor'' hingegen kann aus einem ``string'', ``list'', ``variable'', ``result'' oder wiederum einer ``expr'' welche in Klammern steht bestehen.

\begin{grammar}
<nop> ::= `nop'
\end{grammar}

\begin{grammar}
<variable> ::= <NAME>
\alt <variable> `.' <NAME>
\alt <variable> `[' <expr> `]'
\end{grammar}

\begin{grammar}
<result> ::= <variable> `(' <list\_inner> `)'
\end{grammar}

\begin{grammar}
<print> ::= `print' <expr>
\end{grammar}